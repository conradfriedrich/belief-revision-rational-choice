\documentclass{article} \usepackage[utf8]{inputenc} \title{} \author{Conrad Friedrich
\\ MCMP, LMU Munich \\ 500,000 words} \date{\today} \usepackage{setspace}

\usepackage[backend=biber,authordate,
ibidtracker=context,natbib,doi=false,isbn=false,url=false]{biblatex-chicago}
\addbibresource{/home/gilbert/Documents/bibliography/references.bib}

\begin{document}

\maketitle
\section{Introduction}
This paper discusses Hans Rott' claim that principles of reasoning and belief
are subsumed under and derivable from what may be called principles of practical
reason and, in addition, Erik Olsson's reply. I argue that Olsson's criticism is in part
successful, but Sven Ove Hansson makes an even stronger point. To this end, I
first reconstruct a fragment of Rott's unifying framework and explain the
logical and epistemological systems involved. Then I look in detail
at Olsson's philosophical response. I conclude with a critical discussion.

The first part of this paper focusses on select elements of the formal systems Hans Rott is unifying. I
describe aspects belief revision theory, a study of rational changes of belief, thus a
theory situated between formal epistemology and symbolic artificial
intelligence. Next, I give an exposition of a certain type non-monotonic logic, a type of logic
that allows for defeasible inferences, i.e. inferences which might be
invalidated by further information. I then present results already preceding
Rott's work, in which both systems just mentioned are shown to behave very similarly. 
Next, I give a brief description of the theory of rational choice via choice functions,
and finally describe how Rott shows both belief revision and non-monotonic logic
to be derivable as choice functions.

The second part of this paper accepts the formal systems as given and evaluates
Rott's philosophical claim regarding the unity of practical and theoretical
reason. To this end, I look at Erik Olsson's criticism and discuss his arguments
in turn. I argue that Hansson's brief critique points to a deeper problem than
Olsson addresses, though.  


\section{Belief Revision, Non-monotonic Logic, and Rational Choice}
Belief revision if the form of the most prominent AGM \parencite{alchourron85_logic_theor_chang} defines criteria for a rational agent on how to update her
beliefs when learning evidence. Crucially, this evidence might be conflicting
with what the agent already beliefs. For example, you are about to pay your bill
at a restaurant and, to your great surprise, there is no money in your wallet
\parencite[p.~48]{gaerdenfors88_knowl}.

In this framework, the agent's epistemic state is modeled by a set of sentences
from a standard propositional language \(\mathcal{L}\) closed under the boolean
operations \(\land, \lor, \neg, \rightarrow\). A consistent set closed under a
classical consequence relation \(\vdash\) is called \emph{belief set}. The three
key operations on a belief set are expansion, contraction and revision, each
with a single sentence. This sentence represents external input, e.g. the
content of a testimony of a trusted interlocutor. Expansion is only employed in
case the added belief is consistent with the belief set, such that it is a
special case of revision. Contraction and revision are interdefinable via two
postulates called Levi identity and Harper identity, defining contraction in
terms of revision and \emph{vice versa}, respectively.

Let us look then at the revision function from a sentence \(A\) and a belief set \(K\), denoted \(K^*A\). What constitutes a rational revision? Following Quine`s `maxim of minimum mutilation' \parencite[p.~72]{rott01_chang_choic_infer}, the central postulates for belief revision aim at conserving as much as possible from \(K\) in the new set \(K^*A\) while incorporating the new evidence \(A\).\textcite{gaerdenfors88_knowl} describes six basic, minimal or elementary postulates, and two additional ones. In the interest of brevity, I will skip listing them and refer to the numerous publications agreeing on these principles, e.g. \textcite[p.~57]{gaerdenfors88_knowl, brewka97_nonmon}, \textcite{hansson17_logic_belief_revis}. As an example, this postulates states the success of the revision:

\begin{align}
  \label{eq:k2}
  \tag{(K^*2)} 
  A \in K^*A
\end{align}

\begin{equation}
  \nonumber
  \label{eq:k2}
  \tag{(K^*2)} 
  A \in K^*A
\end{equation}
The postulates are designed are designed to be intuitive and confirm to common sense (Cite). Importantly, they constrain the set of rational revision functions by way of logical (set-theoretical) considerations, but do not necessarily determine a revision uniquely \parencite[p.~53]{gaerdenfors88_knowl}.  

\section{Erik Olsson's Critique}
\section{Discussion and Conclusion}
\nocite{rott01_chang_choic_infer,olsson03_belief_revis_ration_choic_unity_reason,gaerdenfors88_knowl,gaerdenfors94_nonmon_infer_based_expec,alchourron85_logic_theor_chang,makinson91_relat,sen93_inter_consis_choic,kraus90_nonmon_reason_prefer_model_cumul_logic,hansson17_logic_belief_revis, ortner11_mechan_induc,brewka97_nonmon}
\printbibliography
\end{document}